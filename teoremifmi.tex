\documentclass[a4paper,12pt]{article}
\usepackage{amsfonts}
\usepackage{amsmath}

\begin{document}
\title{Teoremi di Fondamenti Matematici per l'Informatica}
\author{Carlo Ramponi}
\date{\today}
\maketitle

\newpage

\newcommand{\N}{\mathbb{N}}
\newcommand{\Z}{\mathbb{Z}}

\newcommand{\teorema}[3]{
	\section{#1}
	\subsection*{Enunciato}
	#2
	\subsection*{Dimostrazione}
	#3
	\clearpage
}

\tableofcontents

\clearpage

\teorema{L'ordinamento dei numeri naturali è un buon ordinamento} 
{
	L'ordinamento dei numeri naturali è un buon ordinamento
}
{
Supponiamo che l'insieme $ A \subseteq \N $ non abbia minimo e proviamo che allora $ A = \emptyset $. Chiamiamo $ B $ il suo complementare ($ B = \N \setminus A $) e dimostriamo per induzione che \[ \forall n \in \N \quad \{0, 1, ..., n\} \subseteq B \]
\begin{itemize}
	\item $ 0 \notin A  $, altrimenti ne sarebbe il minimo, quindi $ 0 \in B $ e pertanto $ \{0\} \subseteq B $.
	\item Supponiamo che $\{0, 1, ..., n\} \subseteq B$, allora $0, 1, ..., n \notin A$ e quindi $n+1 \notin A$, altrimenti ne sarebbe il minimo, ma allora $n+1 \in B$ e pertanto $\{0, 1, ..., n, n+1\} \subseteq B$.\\
\end{itemize}
Per il principio di induzione di prima forma un insieme con queste proprietà coincide con quello dei numeri naturali ($B = \N$) e quindi $A = \emptyset$
}

\teorema{Il principio di induzione (seconda forma)}
{
	Sia $P(n)$ una famiglia di proposizioni indicate su $\N$ e si supponga che
	\begin{enumerate}
		\item $P(0)$ sia vera
		\item $\forall n > 0 (P(k) vera \forall k < n) \Rightarrow P(n) vera$
	\end{enumerate}
	allora $P(n)$ è vera $\forall n \in \N$
}
{
	Sia $A = \{n \in \N | P(n)$ non è vera $\}$, e supponiamo per assurdo che $A \neq \emptyset$.\\
	Allora per la proprietà di buon ordinamento $A$ ha minimo $n$.\\
	Chiaramente $n \neq 0$ in quanto $P(0)$ è vera per ipotesi.\\
	Inoltre se $k < n$ allora $k \notin A$ in quanto $n = \min A$, ma allora dalla (2) segue che $P(n)$ è vera e quindi $n \notin A$, contraddicendo il fatto che $n \in A$.
}

\teorema{La divisione euclidea (esistenza e unicità)}
{
	Siano $n, m \in \Z$ con $m \neq 0$, allora esistono unici $q, r \in \Z$ tali che
	\[ \begin{cases}
			n = mq + r \\
			0 \leq r < |m|
		\end{cases} \]
}
{
\begin{itemize}
	\item \textbf{Esistenza} Supponiamo dapprima che $n, m \in \N$, ed usiamo il principio di induzione della seconda forma su $n$.
		\begin{itemize}
			\item Se $n = 0$ basta prendere $q = 0$ e $r = 0$.
			\item Supponiamo $n > 0$ e che la tesi sia vera $\forall k < n$. Se $n < m$ basta prendere $q = 0$ e $r = n$, altrimenti sia $k = n - m$, dato che $m\neq 0$, $0 < k < n$, quindi per ipotesi di induzione esistono $q, r \in \N$ tali che
			\[ \begin{cases}
				k = mq + r \\
				0 \leq r < |m|
			\end{cases} \]
		\end{itemize}
		ma allora $n = k + m = mq + r + m = (q + 1)m + r$. \\
		Supponiamo ora $n < 0$ e $m > 0$. Allora $-n > 0$ e quindi per il caso precedente si ha che esistono $q, r \in \Z$ tali che $-n = mq + r$ e $0 \leq r < m = |m|$. E quindi $n = m(-q) - r$. Se $r = 0$ abbiamo finito, se invece $0 < r < m$ allora $0 < m - r < m = |m|$ e $n = m(-q) - r = m(-q) -m +m -r = m(-1 -q) + (m-r)$.\\
		Sia infine $m < 0$ allora $-m > 0$, quindi per i due casi precedenti $\exists q, r \in \Z$ tali che $n = (-m)q + r = m(-q) + r$ con $0 \leq r < -m = |m|$
	\item \textbf{Unicità} Supponiamo che $n = mq + r$ e $n = mq' + r'$ con $0 \leq r, r' < m$. \\
		Supponiamo che $r' \geq r$, allora $m(q - q') = r' - r$ e quindi passando ai moduli si ha $|m||q - q'| = |r' - r| = r' - r < |m|$, da cui $0 \leq |q - q'| < 1$ e quindi $|q - q'| = 0$ ovvero $q = q'$.\\
		Ma allora da $mq + r = mq' + r'$ segue che anche $r = r'$.
\end{itemize}
}

\teorema{Codifica dei natuali in base maggiore o uguale a 2}
{
	\textbf{Definizione} Sia $b \in \N$, diremo che $n \in \N$ è rappresentabile in base $b$ se esistono numeri $\epsilon_{0}, \epsilon_{1}, ..., \epsilon_{k} \in I_{b} = \{0, 1, ..., b-1\}$ tali che $n = \epsilon_{0} + \epsilon_{1}b + \epsilon_{2}b^{2} + ... + \epsilon_{k}b^{k}$.
	\\\\
	Sia $b \in \N, b \geq 2$. Allora ogni $n \in \N$ è rappresentabile in modo unico in base $b$. Ossia esiste una successione $\{\epsilon_{i}\}_{i \in \N}$ tale che:
	\begin{enumerate}
		\item $\{\epsilon_{i}$ è definitivamente nulla ($\exists i_{0} \in \N : \epsilon_{i} = 0 \quad \forall i > i_{0}$)
		\item $\epsilon_{i} \in I_{b}$ (ossia $0 \leq \epsilon_{i} < b$) per ogni $i \in \N$
		\item $n = \sum_{i = 0}^{\infty} \epsilon_{i}b^i$
	\end{enumerate}
	e se $\{\epsilon_{i}'\}_{i \in \N}$ è un'altra tale successione, allora $\epsilon_{i} = \epsilon_{i}' \quad \forall i \in \N$
}
{
	\textbf{Esistenza} per induzione su $ n $.
	\begin{enumerate}
		\item Se $ n = 0 $ basta prendere $ \epsilon_{i} = 0 \quad \forall i \in \N $.
		\item Supponiamo ora $ n > 0 $ e che la tesi sia vera per ogni $ k < n $.\\
		Siano $ q, r $ tali che $ n = bq + r $ con $ 0 \leq r < b $. Dato che $ b \geq 2 $ si ha che $ 0 \leq q < bq \leq bq +r = n $ e quindi per l'ipotesi di induzione esiste una successione definitivamente nulla $ \{\delta_{i}\}_{i \in \N} $, costituita da interi tali che $ 0 \leq \delta_{i} < b \quad \forall i \in \N $ e tale che $ q = \sum_{i = 0}^{\infty} \delta_{i}b^{i} $. Ma allora
		\[ n = bq + r = b \sum_{i = 0}^\infty \delta_{i}b^i + r = \sum_{i = 0}^\infty \delta_{i}b^{i+1} + r = \sum_{i = 1}^\infty \delta_{i-1}b^i + r = \sum_{i = 0}^\infty \epsilon_{i}b^i \]
		dove si è posto $ \epsilon_{0} = r $ e $ \epsilon_{i} = \delta_{i-1} \quad \forall i > 0 $.\\
		La successione $ \{ \epsilon_{i} \} $ è definitivamente nulla, dato che lo è $ \{\delta_{i}\} $ ed inoltre $ 0 \leq \epsilon_{i} = \delta_{i-1} < b \quad \forall i > 0 $ e $ 0 \leq \epsilon_{0} = r < b $.
	\end{enumerate}
	\textbf{Unicità} per induzione su $ n $.
	\begin{enumerate}
		\item Se $ n = 0 = \sum_i \epsilon_ib^i $ allora ogni addendo della somma, essendo non negativo, deve essere nullo e quindi $ \epsilon_{i} = 0 \quad \forall i \in \N $
		\item Supponiamo ora $ n > 0 $ e che l'espressione in base $ b $ sia unica per tutti i numeri $ k < n $. Sia $ n $ tale che $ n = \sum_{i = 0}^{\infty} \epsilon_ib^i = \sum_{i = 0}^{\infty} \epsilon_i'b^i $, allora possiamo scrivere
		\[ n = b\sum_{i = 1}^\infty \epsilon_ib^{i-1} + \epsilon_{0} = b\sum_{i = 1}^\infty \epsilon_i'b^{i-1} + \epsilon_{0}' \]
		ma per l'unicità della divisione euclidea si ha che $ \epsilon_{0} = \epsilon_{0}' $ e $ q = \sum_{i = 1}^\infty \epsilon_ib^{i-1} = \sum_{i = 1}^\infty \epsilon_i'b^{i-1} $. Come prima $ q < n $ e quindi per ipotesi induttiva si ha anche che $ \epsilon_i = \epsilon_i' \quad \forall i \geq 1 $
	\end{enumerate}
}

\teorema{Il massimo comun divisore}{
	\textbf{Definizione} Dati due interi $ n, m \in \Z $ non entrambi nulli, si dice che $ d $ è un \textit{massimo comun divisore tra n e m} se:
	\begin{enumerate}
		\item $ d | n $ e $ d | m $ \hfill ( è un divisore )
		\item Se $ c | n $ e $ c | m $ allora $ c | d $ \hfill ( è il massimo )
	\end{enumerate}
	\textbf{Proposizione} Se $ d $ e $ d' $ sono due \textit{massimi comun divisori tra n ed m} allora $ d' = \pm d $.\\
	\textbf{Dimostrazione} $ d $ è un divisore comune di $ n $ e $ m $, quindi poichè $ d' $ è un massimo comun divisore di ha che $ d | d' $. Scambiando i ruoli di $ d $ e $ d' $ si ha allora che anche $ d' | d $ e quindi si ha che $ d' = \pm d $.\\
	\textbf{Definizione} Diremo che $ d $ è il massimo comun divisore di $ n $ e $ m $ se è un massimo comun divisore positivo. La proposizione precedente ci garantisce che se esiste un massimo comun divisore esso è unico.\\\\
	Dati due numeri $ n, m \in \Z $ non entrambi nulli, allora esiste il \textit{massimo comun divisore tra n ed m.}
}{
	\textbf{Esistenza} Si consideri l'insieme \[ S = \{s \in \Z | s > 0, \exists x, y \in \Z : s = nx + my\} \]
	$ S \neq \emptyset $ dato che $ nn + mm > 0 $ (visto che n ed m non sono entrambi nulli).\\
	Sia ora \[ d = nx + my = \min S \]
	dimostriamo che $ d $ è il massimo comun divisore:\\
	Se $ c | n $ e $ c | m $ allora $ n = ck $ e $ m = ch $, quindi $  d = nx + my = ckx + chy = c(kx+hy) $, ossia $ c | d $.\\
	Dimostriamo ora che $ d | n $:\\
	consideriamo la divisione euclidea tra $ n $ e $ d $, ossia $ n = dq + r $ con $ 0 \leq r < d $, se $ r > 0 $, allora $ r = n-dq = n - (nx + my)q = n(1 - qx) + (-m)y \in S $. Ciò è assurdo perchè $ r < d $ e $ d = \min S $. Quindi $ r = 0 $ ossia $ d | n $. In modo del tutto analogo si prova che $ d | m $.
}

\teorema{Il minimo comune multiplo}{
		\textbf{Definizione} Dati due interi $ n, m \in \Z $ si dice che $ M $ è un \textit{minimo comune multiplo di n ed m} se:
		\begin{enumerate}
			\item $ n|M $ e $ m|M $ \hfill ( è un multiplo )
			\item se $ n | c $ e $ m | c $ allora $ M|c $ \hfill ( è il minimo )
		\end{enumerate}
	Come nel caso del massimo comun divisore di dimostra che due minimi comuni multipli sono uguali a meno del segno e quindi si chiama \textit{il minimo comune multiplo} quello positivo (è quindi unico)\\\\
	Siano $ n, m \in \Z $ non entrambi nulli, allora esiste il \textit{minimo comune multiplo tra n e m}.
}{
	\textbf{Esistenza} Sia
	\[ M = \frac{nm}{(n, m)} = n'm'(n, m) \]
	dove si è posto
	\[
		\begin{cases}
			n = n'(n, m)\\
			m = m'(n, m)
		\end{cases}
	\]
	Chiaramente allora $ M = nm' = n'm $ e quindi $ n|M $ e $ m | M $.\\
	Se $ n|c $ e $ m|c $ allora $ (n, m) | c $ e quindi posto $ c = c'(n, m) $ si ha che $ n' | c' $ e $ m' | c' $. Dato che $ (n', m') = 1 $, si ha che $ n'm' | c' $ e quindi che $ M = n'm'(n, m) | c'(n, m) = c $.
}

\teorema{Teorema fondamentale dell'aritmetica}{
	Per ogni $ n \in \Z, n\geq 2 $ esistono numeri primi $ p_1, p_2, ..., p_k > 0 $ tali che $ n = p_1p_2...p_k $\\
	Se anche $ q_1, q_2, ..., q_h $ sono numeri primi positivi tali che $ n = q_1q_2 ... q_h $, allora esiste una bigezione $ \sigma : \{1, 2, ..., h\} \rightarrow \{1, 2, ..., k\} $ tale che $ q_i = p_{\sigma(i)} $.\\
	In altre parole, ogni intero maggiore di 1 si scrive \textbf{in modo unico}, a meno dell'ordine, come \textbf{prodotto di numeri primi positivi}.
}{
	\textbf{Esistenza.} Procediamo per induzione su $ n $:\\
	\begin{enumerate}
		\item Se $ n = 2 $ non c'è nulla da dimostrare in quanto 2 è primo.
		\item Supponiamo $ n > 2 $ e che la tesi sia vera per ogni $ k < n $:\\
			Se $ n $ è primo non c'è nulla da dimostrare,\\
			se $ n $ non è primo allora esistono due numeri $ d_1, d_2 $ con $ 1 < d_1, d_2 < n $ tali che $ n = d_1d_2 $.\\
			Per ipotesi di induzione esistono dei numeri primi positivi tali che $ d_1 = p_1p_2 ... p_{k_1} $ e $ d_2 = q_1 q_2 ... q_{k_2} $,\\
			ma allora $ n = p_1p_2 ... p_{k_1}q_1q_2 ... q_{k_2} $ è prodotto di numeri primi positivi.
	\end{enumerate}
	\textbf{Unicità.} Sia $ n = p_1 ... p_k = q_1 ... q_h $ con $ p_i $ e $ q_j $ numeri primi positivi e $ k \leq h $. Procediamo per induzione su $ k $:
	\begin{enumerate}
		\item Se $ k = 1 $ allora $ n = p_1 = q_1 ... q_h $, quindi $ q_j | p_1 \quad \forall j $, e dato che $ p_1 $ è primo $ q_j = p_1 \quad \forall j $. Se fosse  $ h > 1 $ si avrebbe $ n = q_1 ... q_h \geq q_1q_2 = p_1^2 > p_1 = n $ e questo è assurdo, e quindi $ h = 1 $ e $ q_1 = p_1 $.
		\item Sia $ k > 1 $, allora $ p_k | n = q_1 ... q_h $, quindi esiste un $ j $ tale che $ p_k | q_j $.\\
			Dato che sia $ p_k $ che $ q_j $ sono primi positivi, allora $ p_k = q_j $. Ma allora $ p_1 ... p_{k-1} = q_1 ... q_{j-1}q_{j+1} ... q_h $, per ipotesi di induzione possiamo allora dire che le due fattorizzazioni hanno lo stesso numero di elementi, ossia $ k-1 = h-1 $, e che esiste una bugezione $ \delta : \{1, ..., j-1, j+1, ..., k\} \rightarrow \{1, ..., k-1\} $ tale che $ q_i = p_{\delta(i)} \quad \forall i $. Definendo allora $ \sigma : \{1, 2, ..., k\} \rightarrow \{1, 2, ..., k\} $ tale che
			\[ \sigma(i) = \begin{cases}
				k & \text{se } i = j\\
				\delta(i) & \text{se } i \neq j
			\end{cases} \]
			si ottiene una bigezione tale che $ q_i = p_{\sigma(i)} \quad \forall i $.
	\end{enumerate}
}

\teorema{Il Teorema Cinese del resto}{
	Il sistema di congruenze:
	\[ \begin{cases}
		x \equiv a \quad \mod n\\
		x \equiv b \quad \mod m
	\end{cases} \]
	ha soluzione se e solo se $ (n, m) | b-a $.\\
	Se $ c $ è una soluzione del sistema, allora gli elementi di $ [c]_{[n, m]} $ sono \textbf{tutte} e \textbf{sole} le soluzioni del sistema. (i.e. le soluzioni del sistema sono tutte e sole della forma $ c + k[n, m] $ al variare di $ k \in \Z $).
}{
	Sia $ c $ una soluzione del sistema, allora $ \exists h, k \in \Z $ tali che $ c = a+hn = b+km $ e quindi $ a - b = km - hn $.\\
	Ma allora dal fatto che $ (n, m) | n $ e $ (n, m) | m $ si ha che $ (n, m) | a - b $.\\
	Viceversa, supponiamo che $ (n, m) | a - b $, allora, per quanto visto in precedenza, $ \exists h, k \in \Z $ tali che $ a - b = hn + km $. Ma allora $ a - hn = b+kn $, detto quindi $  c = a-hn = b + km $, si ha evidentemente che $ c $ risolve entrambe le congruenze.\\
	Sia $ S = \{x \in \Z \  | \  x \text{ risolve il sistema}\} $.  Dobbiamo provare che se $ c $ è una soluzione del sistema allora $ S = [c]_{[n, m]} $.
	\begin{itemize}
		\item $\mathbf{ S \subseteq [c]_{[n, m]} }$. Sia $ c' $ un'altra soluzione del sistema, allora $ c = a+hn = b+km $ e $ c' = a+h'n = b+k'm $ e quindi sottraendo si ha:
		\[ \begin{matrix}
			c - c' & = & a+hn - a-h'n = (h-h')n & \Rightarrow & n \ | \ (c-c') \\
			c - c' & = & a+km - a-k'm = (k-k')m & \Rightarrow & m \ | \ (c-c')
		\end{matrix}
		 \]
		 Ma allora $ [n, m] \ | \ c - c' $, ossia $ c' \equiv c \mod [n, m] $ ovvero $ c' \in [c]_{[n, m]} $.
		 \item $\mathbf{ [c]_{[n, m]}  \subseteq S }$. Sia $ c' \in [c]_{[n, m]}] $, ovvero $ c' = c+h[n, m] $. Dal fatto che $ c \equiv a \mod n $ e che $ h[n, m] = \equiv 0 \mod n $ segue che $ c' = c + h[n, m] \equiv a \mod n $. In modo analogo si ha che $  c' \equiv b \mod m $ e quindi che $ c' \in S $.
	\end{itemize}
}

\end{document}

















