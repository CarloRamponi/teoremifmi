\documentclass[a4paper,12pt]{article}
\usepackage{amsfonts}
\usepackage{amsmath}

\begin{document}
\title{Teoremi di Fondamenti Matematici per l'Informatica}
\author{Carlo Ramponi}
\date{\today}
\maketitle

\newpage

\tableofcontents

\clearpage

\section{L'ordinamento dei numeri naturali è un buon ordinamento} 
\subsection*{Enunciato}
L'ordinamento dei numeri naturali è un buon ordinamento
\subsection*{Dimostrazione}
Supponiamo che l'insieme $ A \subseteq \mathbb{N} $ non abbia minimo e proviamo che allora $ A = \emptyset $. Chiamiamo $ B $ il suo complementare ($ B = \mathbb{N} \setminus A $) e dimostriamo per induzione che \[ \forall n \in \mathbb{N} \quad \{0, 1, ..., n\} \subseteq B \]
\begin{itemize}
	\item $ 0 \notin A  $, altrimenti ne sarebbe il minimo, quindi $ 0 \in B $ e pertanto $ \{0\} \subseteq B $.
	\item Supponiamo che $\{0, 1, ..., n\} \subseteq B$, allora $0, 1, ..., n \notin A$ e quindi $n+1 \notin A$, altrimenti ne sarebbe il minimo, ma allora $n+1 \in B$ e pertanto $\{0, 1, ..., n, n+1\} \subseteq B$.\\
\end{itemize}
Per il principio di induzione di prima forma un insieme con queste proprietà coincide con quello dei numeri naturali ($B = \mathbb{N}$) e quindi $A = \emptyset$

\clearpage
\section{Il principio di induzione (seconda forma)}
\subsection*{Enunciato}
Sia $P(n)$ una famiglia di proposizioni indicate su $\mathbb{N}$ e si supponga che
\begin{enumerate}
	\item $P(0)$ sia vera
	\item $\forall n > 0 (P(k) vera \forall k < n) \Rightarrow P(n) vera$
\end{enumerate}
allora $P(n)$ è vera $\forall n \in \mathbb{N}$

\subsection*{Dimostrazione}
Sia $A = \{n \in \mathbb{N} | P(n)$ non è vera $\}$, e supponiamo per assurdo che $A \neq \emptyset$.\\
Allora per la proprietà di buon ordinamento $A$ ha minimo $n$.\\
Chiaramente $n \neq 0$ in quanto $P(0)$ è vera per ipotesi.\\
Inoltre se $k < n$ allora $k \notin A$ in quanto $n = \min A$, ma allora dalla (2) segue che $P(n)$ è vera e quindi $n \notin A$, contraddicendo il fatto che $n \in A$.

\clearpage

\section{La divisione euclidea (esistenza e unicità)}
\subsection*{Enunciato}
Siano $n, m \in \mathbb{Z}$ con $m \neq 0$, allora esistono unici $q, r \in \mathbb{Z}$ tali che
\[ \begin{cases}
		n = mq + r \\
		0 \leq r < |m|
	\end{cases} \]
	
\subsection*{Dimostrazione}
\begin{itemize}
	\item \textbf{Esistenza} Supponiamo dapprima che $n, m \in \mathbb{N}$, ed usiamo il principio di induzione della seconda forma su $n$.
		\begin{itemize}
			\item Se $n = 0$ basta prendere $q = 0$ e $r = 0$.
			\item Supponiamo $n > 0$ e che la tesi sia vera $\forall k < n$. Se $n < m$ basta prendere $q = 0$ e $r = n$, altrimenti sia $k = n - m$, dato che $m\neq 0$, $0 < k < n$, quindi per ipotesi di induzione esistono $q, r \in \mathbb{N}$ tali che
			\[ \begin{cases}
				k = mq + r \\
				0 \leq r < |m|
			\end{cases} \]
		\end{itemize}
		ma allora $n = k + m = mq + r + m = (q + 1)m + r$. \\
		Supponiamo ora $n < 0$ e $m > 0$. Allora $-n > 0$ e quindi per il caso precedente si ha che esistono $q, r \in \mathbb{Z}$ tali che $-n = mq + r$ e $0 \leq r < m = |m|$. E quindi $n = m(-q) - r$. Se $r = 0$ abbiamo finito, se invece $0 < r < m$ allora $0 < m - r < m = |m|$ e $n = m(-q) - r = m(-q) -m +m -r = m(-1 -q) + (m-r)$.\\
		Sia infine $m < 0$ allora $-m > 0$, quindi per i due casi precedenti $\exists q, r \in \mathbb{Z}$ tali che $n = (-m)q + r = m(-q) + r$ con $0 \leq r < -m = |m|$
	\item \textbf{Unicità} Supponiamo che $n = mq + r$ e $n = mq' + r'$ con $0 \leq r, r' < m$. \\
		Supponiamo che $r' \geq r$, allora $m(q - q') = r' - r$ e quindi passando ai moduli si ha $|m||q - q'| = |r' - r| = r' - r < |m|$, da cui $0 \leq |q - q'| < 1$ e quindi $|q - q'| = 0$ ovvero $q = q'$.\\
		Ma allora da $mq + r = mq' + r'$ segue che anche $r = r'$.
\end{itemize}

\clearpage

\section{Codifica dei natuali in base maggiore o uguale a 2}
\textbf{Definizione} Sia $b \in \mathbb{N}$, diremo che $n \in \mathbb{N}$ è rappresentabile in base $b$ se esistono numeri $\epsilon_{0}, \epsilon_{1}, ..., \epsilon_{k} \in I_{b} = \{0, 1, ..., b-1\}$ tali che $n = \epsilon_{0} + \epsilon_{1}b + \epsilon_{2}b^{2} + ... + \epsilon_{k}b^{k}$.

\subsection*{Enunciato}
Sia $b \in \mathbb{N}, b \geq 2$. Allora ogni $n \in \mathbb{N}$ è rappresentabile in modo unico in base $b$. Ossia esiste una successione $\{\epsilon_{i}\}_{i \in \mathbb{N}}$ tale che:
\begin{enumerate}
	\item $\{\epsilon_{i}$ è definitivamente nulla ($\exists i_{0} \in \mathbb{N} : \epsilon_{i} = 0 \quad \forall i > i_{0}$)
	\item $\epsilon_{i} \in I_{b}$ (ossia $0 \leq \epsilon_{i} < b$) per ogni $i \in \mathbb{N}$
	\item $n = \sum_{i = 0}^{\infty} \epsilon_{i}b^i$
\end{enumerate}
e se $\{\epsilon_{i}'\}_{i \in \mathbb{N}}$ è un'altra tale successione, allora $\epsilon_{i} = \epsilon_{i}' \quad \forall i \in \mathbb{N}$

\subsection*{Dimostrazione}
\textbf{Esistenza} per induzione su $ n $.
\begin{enumerate}
	\item Se $ n = 0 $ basta prendere $ \epsilon_{i} = 0 \quad \forall i \in \mathbb{N} $.
	\item Supponiamo ora $ n > 0 $ e che la tesi sia vera per ogni $ k < n $.\\
	Siano $ q, r $ tali che $ n = bq + r $ con $ 0 \leq r < b $. Dato che $ b \geq 2 $ si ha che $ 0 \leq q < bq \leq bq +r = n $ e quindi per l'ipotesi di induzione esiste una successione definitivamente nulla $ \{\delta_{i}\}_{i \in \mathbb{N}} $, costituita da interi tali che $ 0 \leq \delta_{i} < b \quad \forall i \in \mathbb{N} $ e tale che $ q = \sum_{i = 0}^{\infty} \delta_{i}b^{i} $. Ma allora
	\[ n = bq + r = b \sum_{i = 0}^\infty \delta_{i}b^i + r = \sum_{i = 0}^\infty \delta_{i}b^{i+1} + r = \sum_{i = 1}^\infty \delta_{i-1}b^i + r = \sum_{i = 0}^\infty \epsilon_{i}b^i \]
	dove si è posto $ \epsilon_{0} = r $ e $ \epsilon_{i} = \delta_{i-1} \quad \forall i > 0 $.\\
	La successione $ \{ \epsilon_{i} \} $ è definitivamente nulla, dato che lo è $ \{\delta_{i}\} $ ed inoltre $ 0 \leq \epsilon_{i} = \delta_{i-1} < b \quad \forall i > 0 $ e $ 0 \leq \epsilon_{0} = r < b $.
\end{enumerate}
\textbf{Unicità} per induzione su $ n $.
\begin{enumerate}
	\item Se $ n = 0 = \sum_i \epsilon_ib^i $ allora ogni addendo della somma, essendo non negativo, deve essere nullo e quindi $ \epsilon_{i} = 0 \quad \forall i \in \mathbb{N} $
	\item Supponiamo ora $ n > 0 $ e che l'espressione in base $ b $ sia unica per tutti i numeri $ k < n $. Sia $ n $ tale che $ n = \sum_{i = 0}^{\infty} \epsilon_ib^i = \sum_{i = 0}^{\infty} \epsilon_i'b^i $, allora possiamo scrivere
	\[ n = b\sum_{i = 1}^\infty \epsilon_ib^{i-1} + \epsilon_{0} = b\sum_{i = 1}^\infty \epsilon_i'b^{i-1} + \epsilon_{0}' \]
	ma per l'unicità della divisione euclidea si ha che $ \epsilon_{0} = \epsilon_{0}' $ e $ q = \sum_{i = 1}^\infty \epsilon_ib^{i-1} = \sum_{i = 1}^\infty \epsilon_i'b^{i-1} $. Come prima $ q < n $ e quindi per ipotesi induttiva si ha anche che $ \epsilon_i = \epsilon_i' \quad \forall i \geq 1 $
\end{enumerate}

\end{document}

















