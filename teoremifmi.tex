\documentclass[a4paper,12pt]{article}
\usepackage{makeidx}
\usepackage{amsfonts}
\usepackage{amsmath}
\makeindex
\begin{document}
\title{Teoremi di Fondamenti Matematici per l'Informatica}
\author{Carlo Ramponi}
\date{\today}
\maketitle

\newpage
\section{L'ordinamento dei numeri naturali è un buon ordinamento} 
\subsection*{Enunciato}
L'ordinamento dei numeri naturali è un buon ordinamento
\subsection*{Dimostrazione}
Supponiamo che l'insieme $ A \subseteq \mathbb{N} $ non abbia minimo e proviamo che allora $ A = \emptyset $. Chiamiamo $ B $ il suo complementare ($ B = \mathbb{N} \setminus A $) e dimostriamo per induzione che \[ \forall n \in \mathbb{N} \quad \{0, 1, ..., n\} \subseteq B \]
\begin{itemize}
	\item $ 0 \notin A  $, altrimenti ne sarebbe il minimo, quindi $ 0 \in B $ e pertanto $ \{0\} \subseteq B $.
	\item Supponiamo che $\{0, 1, ..., n\} \subseteq B$, allora $0, 1, ..., n \notin A$ e quindi $n+1 \notin A$, altrimenti ne sarebbe il minimo, ma allora $n+1 \in B$ e pertanto $\{0, 1, ..., n, n+1\} \subseteq B$.\\
\end{itemize}
Per il principio di induzione di prima forma un insieme con queste proprietà coincide con quello dei numeri naturali ($B = \mathbb{N}$) e quindi $A = \emptyset$

\clearpage
\section{Il principio di induzione (seconda forma)}
\subsection*{Enunciato}
Sia $P(n)$ una famiglia di proposizioni indicate su $\mathbb{N}$ e si supponga che
\begin{enumerate}
	\item $P(0)$ sia vera
	\item $\forall n > 0 (P(k) vera \forall k < n) \Rightarrow P(n) vera$
\end{enumerate}
allora $P(n)$ è vera $\forall n \in \mathbb{N}$

\subsection*{Dimostrazione}
Sia $A = \{n \in \mathbb{N} | P(n)$ non è vera $\}$, e supponiamo per assurdo che $A \neq \emptyset$.\\
Allora per la proprietà di buon ordinamento $A$ ha minimo $n$.\\
Chiaramente $n \neq 0$ in quanto $P(0)$ è vera per ipotesi.\\
Inoltre se $k < n$ allora $k \notin A$ in quanto $n = \min A$, ma allora dalla (2) segue che $P(n)$ è vera e quindi $n \notin A$, contraddicendo il fatto che $n \in A$.

\clearpage

\section{La divisione euclidea (esistenza e unicità)}
\subsection*{Enunciato}
Siano $n, m \in \mathbb{Z}$ con $m \neq 0$, allora esistono unici $q, r \in \mathbb{Z}$ tali che
\[ \begin{cases}
		n = mq + r \\
		0 \leq r < |m|
	\end{cases} \]
	
\subsection*{Dimostrazione}
\begin{itemize}
	\item \textbf{Esistenza} Supponiamo dapprima che $n, m \in \mathbb{N}$, ed usiamo il principio di induzione della seconda forma su $n$.
		\begin{itemize}
			\item Se $n = 0$ basta prendere $q = 0$ e $r = 0$.
			\item Supponiamo $n > 0$ e che la tesi sia vera $\forall k < n$. Se $n < m$ basta prendere $q = 0$ e $r = n$, altrimenti sia $k = n - m$, dato che $m\neq 0$, $0 < k < n$, quindi per ipotesi di induzione esistono $q, r \in \mathbb{N}$ tali che
			\[ \begin{cases}
				k = mq + r \\
				0 \leq r < |m|
			\end{cases} \]
		\end{itemize}
		ma allora $n = k + m = mq + r + m = (q + 1)m + r$. \\
		Supponiamo ora $n < 0$ e $m > 0$. Allora $-n > 0$ e quindi per il caso precedente si ha che esistono $q, r \in \mathbb{Z}$ tali che $-n = mq + r$ e $0 \leq r < m = |m|$. E quindi $n = m(-q) - r$. Se $r = 0$ abbiamo finito, se invece $0 < r < m$ allora $0 < m - r < m = |m|$ e $n = m(-q) - r = m(-q) -m +m -r = m(-1 -q) + (m-r)$.\\
		Sia infine $m < 0$ allora $-m > 0$, quindi per i due casi precedenti $\exists q, r \in \mathbb{Z}$ tali che $n = (-m)q + r = m(-q) + r$ con $0 \leq r < -m = |m|$
	\item \textbf{Unicità} Supponiamo che $n = mq + r$ e $n = mq' + r'$ con $0 \leq r, r' < m$. \\
		Supponiamo che $r' \geq r$, allora $m(q - q') = r' - r$ e quindi passando ai moduli si ha $|m||q - q'| = |r' - r| = r' - r < |m|$, da cui $0 \leq |q - q'| < 1$ e quindi $|q - q'| = 0$ ovvero $q = q'$.\\
		Ma allora da $mq + r = mq' + r'$ segue che anche $r = r'$.
\end{itemize}

\printindex
\end{document}