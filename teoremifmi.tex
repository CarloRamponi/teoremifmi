\documentclass[a4paper,12pt]{article}
\usepackage{makeidx}
\usepackage{amsfonts}
\makeindex
\begin{document}
\title{Teoremi di Fondamenti Matematici per l'Informatica}
\author{Carlo Ramponi}
\date{\today}
\maketitle

\newpage
\section{L'ordinamento dei numeri naturali è un buon ordinamento} 
\subsection*{Enunciato}
L'ordinamento dei numeri naturali è un buon ordinamento
\subsection*{Dimostrazione}
Supponiamo che l'insieme $ A \subseteq \mathbb{N} $ non abbia minimo e proviamo che allora $ A = \emptyset $. Chiamiamo $ B $ il suo complementare ($ B = \mathbb{N} \setminus A $) e dimostriamo per induzione che \[ \forall n \in \mathbb{N} \quad \{0, 1, ..., n\} \subseteq B \]
\begin{itemize}
	\item $ 0 \notin A  $, altrimenti ne sarebbe il minimo, quindi $ 0 \in B $ e pertanto $ \{0\} \subseteq B $.
	\item Supponiamo che $\{0, 1, ..., n\} \subseteq B$, allora $0, 1, ..., n \notin A$ e quindi $n+1 \notin A$, altrimenti ne sarebbe il minimo, ma allora $n+1 \in B$ e pertanto $\{0, 1, ..., n, n+1\} \subseteq B$.\\
\end{itemize}
Per il principio di induzione di prima forma un insieme con queste proprietà coincide con quello dei numeri naturali ($B = \mathbb{N}$) e quindi $A = \emptyset$

\clearpage
\section{Il principio di induzione (seconda forma)}
\subsection*{Enunciato}
Sia $P(n)$ una famiglia di proposizioni indicate su $\mathbb{N}$ e si supponga che
\begin{enumerate}
	\item $P(0)$ sia vera
	\item $\forall n > 0 (P(k) vera \forall k < n) \Rightarrow P(n) vera$
\end{enumerate}
allora $P(n)$ è vera $\forall n \in \mathbb{N}$

\subsection*{Dimostrazione}
Sia $A = \{n \in \mathbb{N} | P(n)$ non è vera $\}$, e supponiamo per assurdo che $A \neq \emptyset$.\\
Allora per la proprietà di buon ordinamento $A$ ha minimo $n$.\\
Chiaramente $n \neq 0$ in quanto $P(0)$ è vera per ipotesi.\\
Inoltre se $k < n$ allora $k \notin A$ in quanto $n = \min A$, ma allora dalla (2) segue che $P(n)$ è vera e quindi $n \notin A$, contraddicendo il fatto che $n \in A$.

\printindex
\end{document}